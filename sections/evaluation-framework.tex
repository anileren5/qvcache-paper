\begin{figure}[h]
    \centering
    \includegraphics[width=\linewidth]{plots/noise_ratio_analysis/neighbor_overlap_analysis.pdf}
    \caption{\textbf{Neighbor overlap under perturbation (GIST, $k=10$).} Intersection of neighbor sets decays sharply, reaching near-zero at a 0.5 noise ratio. }
    \label{fig:overlap-analysis}
\end{figure}

\section{Evaluation Framework}
Real-world vector search workloads exhibit skewed access patterns \cite{298625SmartANN, incrementalivfindexmaintenance}, with spatially close queries recurring (i.e., \textit{temporal-semantic locality}), but existing systems are typically evaluated without accounting for this behavior, making such evaluations inadequate for assessing vector caches. Standard benchmarks execute each query only once and report aggregate metrics such as recall and latency, which is sufficient for evaluating standalone ANN systems but insufficient for understanding cache behavior. 

To address this, we propose a workload generation framework that produces query patterns with \textit{temporal-semantic locality} and tunable parameters. We first partition the base query set into $N_{\text{split}}$ disjoint subsets to model shifts in the regions targeted by queries. Within each subset, we generate perturbed variants for each query. For a query $q$, we sample a random vector $r$ from the dataset and produce  

\[
q' = (1-\eta) \cdot q + \eta \cdot r,
\]  

where $\eta$ controls the noise ratio. This interpolation yields queries that are semantically similar, i.e., spatially close, while remaining distinct, enabling realistic evaluation of vector caches.

To generate recurrence of spatially close queries, we employ a windowed query pattern. Each window consists of perturbed versions of $WINDOW\_SIZE$ base splits. Queries within the window are randomly shuffled and dispatched to the system by search threads, repeating $N_{\text{repeat}}$ times. After each repetition, $stride$ out of $WINDOW\_SIZE$ perturbed splits are replaced with new ones. This process continues until the window reaches the last splits, and the cycle can optionally be repeated $N_{\text{round}}$ times with fresh perturbed copies.

The parameter $WINDOW\_SIZE$ controls the working set size of the workload. $N_{\text{repeat}}$ measures short-term memory, i.e., the ability to capture cache hits within a small time window, while $N_{\text{round}}$ measures long-term memory across multiple cycles. The ratio $stride/WINDOW\_SIZE$ adjusts how quickly the working set evolves. Together, these parameters allow us to generate workloads with varying locality and temporal characteristics, enabling realistic evaluation of vector caches.


\begin{figure*}[t]
    \centering
    \begin{subfigure}[t]{0.48\textwidth}
        \centering
        \begin{tikzpicture}[scale=0.75, transform shape, >=Stealth, font=\small]
    % Define local colors
    \definecolor{c0l}{RGB}{250, 218, 221} \definecolor{c0m}{RGB}{240, 128, 128} \definecolor{c0d}{RGB}{220, 20, 60}
    \definecolor{c1l}{RGB}{200, 240, 200} \definecolor{c1m}{RGB}{144, 238, 144} \definecolor{c1d}{RGB}{34, 139, 34}
    \definecolor{c2l}{RGB}{173, 216, 230} \definecolor{c2m}{RGB}{100, 149, 237} \definecolor{c2d}{RGB}{30, 144, 255}
    \definecolor{c3l}{RGB}{255, 250, 205} \definecolor{c3m}{RGB}{255, 215, 0}   \definecolor{c3d}{RGB}{255, 165, 0}

    % LEVEL 1: Base Queries
    \node[draw, fill=gray!10, minimum width=5cm, minimum height=0.7cm] (base) at (0, 0) {Base Queries in the Dataset};
    
    % LEVEL 2: Split Nodes (S0 - S3)
    \foreach \i/\col in {0/c0l, 1/c1l, 2/c2l, 3/c3l} {
        \node[draw, fill=\col, minimum width=1.0cm, minimum height=0.8cm] (s\i) at ({(\i-1.5)*1.0}, -1.75) {S\i};
    }
    
    % LEVEL 3: Perturbations (C_i,j)
    \def\groupspace{2.5} % horizontal space between groups
    \foreach \i/\cbase in {0/c0, 1/c1, 2/c2, 3/c3} {
        \foreach \j in {0,1,2} {
            \pgfmathsetmacro{\xpos}{(\i-1.5)*\groupspace + (\j-1)*0.65}
            \ifcase\j \colorlet{cc}{\cbase l} \or \colorlet{cc}{\cbase m} \or \colorlet{cc}{\cbase d} \fi
            \node[draw, fill=cc, minimum size=0.6cm] (c\i\j) at (\xpos, -3.8) {\scriptsize C\i,\j};
            
            % Dotted guides from S to middle child
            \ifnum\j=1
                \draw[->, gray, thin, dashed] (s\i.south) -- (c\i\j.north);
            \fi
        }
        
        % Dots between groups
        \ifnum\i>0
            \path (c\the\numexpr\i-1\relax2.east) -- (c\i0.west) node[midway] {\dots};
        \fi
    }

    % Main logic arrows
    \draw[->, dotted, thick] (base.south) -- (0, -1.25) node[midway, right] {\scriptsize $N\_Split=4$};
    
    % Middle label area
    \node[right] at (0.2, -2.8) {\scriptsize $Noise\_Ratio=0.01$};
    \draw[->, dotted, thick] (0,-2.25) -- (0,-3.4);

\end{tikzpicture}

        \caption{Synthesizing Queries with Semantic (Spatial) Locality}
        \label{fig:query-perturbation}
    \end{subfigure}
    \hfill
    \begin{subfigure}[t]{0.48\textwidth}
        \centering
        \resizebox{0.8\textwidth}{!}{
            \begin{tikzpicture}[x=0.8pt,y=0.8pt,yscale=-1,xscale=1]

% --- CONFIGURATION ---
\def\cw{53} % Cell Width
\def\ch{51} % Cell Height
\definecolor{colorC0}{RGB}{246, 174, 174} % Light Red
\definecolor{colorC1}{RGB}{210, 244, 175} % Light Green
\definecolor{colorC2}{RGB}{146, 191, 239} % Light Blue
\definecolor{colorC3}{RGB}{242, 235, 151} % Light Yellow

% --- LEFT GRID (Main Grid) ---
% Row 0
\draw[fill=colorC0] (104, 64) rectangle ++(\cw,\ch) node[midway] {C0,0};
\draw[fill=colorC1] (157, 64) rectangle ++(\cw,\ch) node[midway] {C1,0};
\draw[fill=colorC2] (210, 64) rectangle ++(\cw,\ch) node[midway] {C2,0};
\draw[fill=colorC3] (263, 64) rectangle ++(\cw,\ch) node[midway] {C3,0};

% Row 1
\draw[fill={rgb,255:red,246; green,106; blue,106}] (104, 115) rectangle ++(\cw,\ch) node[midway] {C0,1};
\draw[fill={rgb,255:red,154; green,225; blue,80}]  (157, 115) rectangle ++(\cw,\ch) node[midway] {C1,1};
\draw[fill={rgb,255:red,74;  green,150; blue,239}] (210, 115) rectangle ++(\cw,\ch) node[midway] {C2,1};
\draw[fill={rgb,255:red,242; green,230; blue,43}]  (263, 115) rectangle ++(\cw,\ch) node[midway] {C3,1};

% Lower Blocks (Maintained colors from your original)
\draw[fill={rgb,255:red,121; green,231; blue,12}]  (157, 166) rectangle ++(\cw,\ch) node[midway] {C1,2};
\draw[fill={rgb,255:red,85;  green,168; blue,1}]   (157, 217) rectangle ++(\cw,\ch) node[midway] {C1,3};
\draw[fill={rgb,255:red,8;   green,107; blue,222}] (210, 166) rectangle ++(\cw,\ch) node[midway] {C2,2};
\draw[fill={rgb,255:red,129; green,158; blue,193}] (210, 217) rectangle ++(\cw,\ch) node[midway] {C2,3};
\draw[fill={rgb,255:red,248; green,225; blue,4}]   (263, 166) rectangle ++(\cw,\ch) node[midway] {C3,2};
\draw[fill={rgb,255:red,183; green,164; blue,0}]   (263, 217) rectangle ++(\cw,\ch) node[midway] {C3,3};
\draw[fill={rgb,255:red,184; green,131; blue,232}] (316, 166) rectangle ++(\cw,\ch) node[midway] {C4,0};
\draw[fill={rgb,255:red,118; green,53;  blue,176}] (316, 217) rectangle ++(\cw,\ch) node[midway] {C4,1};

% --- RED FOCUS BOX (Left) ---
\draw [red, line width=2pt] (93, 53) rectangle (327, 126);

% --- SHUFFLE BOX (Top Right) ---
% Now filled completely using colors from the Window (C0, C1, C2, C3)
\foreach \i/\shufcol in {
    0/colorC3, 1/colorC2, 2/colorC1, 3/colorC1, 4/colorC1, 5/colorC2, 
    6/colorC0, 7/colorC0, 8/colorC3, 9/colorC1, 10/colorC0, 11/colorC3, 
    12/colorC0, 13/colorC1, 14/colorC2, 15/colorC2, 16/colorC3, 17/colorC3, 
    18/colorC0, 19/colorC2, 20/colorC1, 21/colorC2, 22/colorC2} 
{
    \draw[fill=\shufcol, draw=black, line width=0.4pt]
        (433 + \i*9.2, 59) rectangle ++(9.5, 51);
}
% Outline for the shuffle bar
\draw [red, line width=1.5pt] (422, 48) rectangle (656, 121);

% --- BOTTOM RIGHT GRID ---
\draw[fill={rgb,255:red,232; green,153; blue,19}] (418, 331) rectangle ++(\cw,\ch) node[midway] {C9,6};
\draw[fill={rgb,255:red,231; green,179; blue,97}] (418, 382) rectangle ++(\cw,\ch) node[midway] {C9,7};
\draw[fill={rgb,255:red,174; green,43;  blue,200}] (471, 331) rectangle ++(\cw,\ch) node[midway] {C8,4};
\draw[fill={rgb,255:red,201; green,136; blue,215}] (471, 382) rectangle ++(\cw,\ch) node[midway] {C8,5};
\draw[fill={rgb,255:red,155; green,155; blue,155}] (524, 331) rectangle ++(\cw,\ch) node[midway] {C9,2};
\draw[fill={rgb,255:red,105; green,105; blue,102}] (524, 382) rectangle ++(\cw,\ch) node[midway] {C9,3};
\draw[fill={rgb,255:red,208; green,55;  blue,73}]  (577, 331) rectangle ++(\cw,\ch) node[midway] {C0,2};
\draw[fill={rgb,255:red,241; green,7;   blue,34}]  (577, 382) rectangle ++(\cw,\ch) node[midway] {C0,3};

% --- TEXT AND ARROWS ---
\node at (210, 30) {\Large WINDOW SIZE = 4};
\draw [dashed, <->] (100, 40) -- (320, 40);

\node [anchor=east] at (65, 110) {\Large N\_REPEAT = 2};
\draw [dashed, <->] (75, 52) -- (75, 165);

\node [anchor=north] at (135, 175) {\Large STRIDE = 1};
\draw [dashed, <->] (102, 172) -- (164, 172);

\node [anchor=north] at (135, 350) {\Large $N\_Split=10, Noise\_Ratio = 0.01$};



\node at (380, 75) {\Large SHUFFLE};
\draw [dashed, ->] (336, 91) -- (416, 91);

\node at (430, 210) {\Large WINDOW SLIDES};
\draw [dashed, -] (352, 103) -- (415, 190);
\draw [dashed, ->] (445, 223) -- (508, 313);

\node at (570, 270) {\Large N\_ROUND};
\node [red] at (590, 290) {\Large +1};
\draw [dashed, ->] (576, 280) -- (576, 323);

\node at (380, 275) {.}; \node at (393, 290) {.}; \node at (405, 305) {.};

\end{tikzpicture}
        }
        \caption{Workload Generation with temporal-semantic Locality}
        \label{fig:sliding-window}
    \end{subfigure}
    \caption{Evaluation framework proposed in this paper for benchmarking vector caches, used to evaluate QVCache.}
    \label{fig:evaluation-framework}
\end{figure*}







